\documentclass[a4paper, 12pt]{article}
\usepackage{color}
\usepackage{graphicx}

\begin{document}

\title{My First Document}
\author{Rishikesh Vanarse}
\date{\today}
\maketitle

\pagenumbering{roman}
\tableofcontents
\newpage
\pagenumbering{arabic}



\section{Introduction}
This is the introduction. The contents of the pdf do not make any sense. This is just a trial for different typesets, images, tables, equations, etc.




\section{Methods}
The method is described as follows:


\subsection{Stage1}
\label{sec1}
The first part of the methods. Some parts of this section are written in \textit{italics} and some parts are in \textbf{bold.}
%Line Comment
Sentence on the same line.
\\Sentence on a new line.
Trial sentence with special characters: Item \#1A\textbackslash642 costs \$8 at a \~{}10\% profit. Citation trial \cite{Birdetal2001}.



\subsection{Stage 2}
The second part of methods.
The package \texttt{color} is used to write things in {\color{red}color}.
Here is a numbered list: 

\begin{enumerate}
\item{Item 1}
\item{Item2}
This item consists of a bulleted list.
\begin{itemize}
\item{Sub-item 1}
\item{Sub-item 2}
\end{itemize}
\item{Item 3}
\end{enumerate}

This section also contains a figure, that may be automatically placed somewhere else. See figure \ref{bdome}. 
%[h] indicates that the figure is approximately here
%[t]=top, [b]=bottom (of page) and [p]=on separate page for all figures
\begin{figure}[h]
\centering
\includegraphics[width=0.3\textwidth]{bdome.png}
\caption{Image of the B-dome}
\label{bdome}
\end{figure}



\subsection{Equations}
This is an equation within the line: $1 + 2 = 3$ and this is an equation that is displayed on its own line: $$1 + 2 = 3$$
A numbered equation can be written as:
\begin{equation}1 + 2 = 3 \end{equation}
Derivation:
\begin{eqnarray*}
something & = & a + a + a + x + x +y +y\\
& = &3a + x + x + y + y\\
& = &3a + 2x + y + y\\
& = &3a + 2x + 2y
\end{eqnarray*}
Using mathematical symbols:
$$x_1 = \frac{-b + \sqrt[2]{b^2 - 4ac}}{2a}$$
$$\sum_{i=1}^n (y_i - y_{i-1})^2$$
$$\Delta = \alpha\int_a^b f(x)dx$$




\subsection{Tables}
\label{tables}
%First define the columns
%Then write each row one by one
\textbf{Without lines}\\
\\
\begin{tabular}{|l|l|l|}
\textbf{Column1} & \textbf{Column2} & \textbf{Column3} \\
Apple & Red & Fruit \\
Banana & Yellow & Fruit\\
Whale & Blue & Fish\\
\end{tabular}
\\
\\


\textbf{With lines}\\
\\
\begin{tabular}{|l|l|l|}
\hline
\textbf{Column1} & \textbf{Column2} & \textbf{Column3} \\
\hline
Apple & Red & Fruit \\
\hline
Banana & Yellow & Fruit\\
\hline
Whale & Blue & Fish\\
\hline \hline
Last row &last row & last row\\
\hline
\end{tabular}





\section{Results}
Here are the results! Refer section \ref{sec1} on page \pageref{sec1}.

\bibliographystyle{plain}
\bibliography{references}

\end{document}

